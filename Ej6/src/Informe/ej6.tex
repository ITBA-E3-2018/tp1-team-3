\documentclass[../../../informe/src/main.tex]{subfiles}
\begin{document}
\section{Ejercicio 6}

Se diseño una ALU de la siguiente manera:
\begin{figure}[H]
\centering
\includegraphics[width=0.5\textwidth]{imagenes/alu.png}
\caption{Implementasión Alu 4 bits} \label{fig=alu}
\end{figure}
 A y B son los registros de entrada, y para seleccionar cada operacion se deve poner el bit en uno de cada mux.

\begin{table}[h]
\begin{center}
\begin{tabular}{|l|l|l|l|l|l|l|l|}
\hline
&A$->$B & CIN &noB$->$B  &  A en la suma & AXor B$->$B & Aor B$->$B &Aand B$->$B\\
\hline \hline
 Suma& 0 & 0 & 0 & 1 & 0&0 &0 \\ \hline
Resta& 0 & 1 & 0 & 1 & 0&0 &0 \\ \hline
 And& 0 & 0 & 0 & 0 & 0&0 &1 \\ \hline
 Or& 0 & 0 & 0 & 0 & 0&1 &0 \\ \hline
 Not& 0 & 0 & 1 & 0 & 0&0 &0 \\ \hline
 Xor& 0 & 0 & 0 & 0 & 1&0 &0 \\ \hline
 Ca2& 0 & 1 & 1& 0& 0&0 &0 \\ \hline
Shl& 1 & 0 & 0 & 1 & 0&0 &0 \\ \hline

\end{tabular}
\caption{Tabla de seleccion de operacion} 
\label{tab=alu op}
\end{center}
\end{table}

\begin{table}[h]
\begin{center}
\begin{tabular}{|l|l|}
\hline
Operación&Opcode\\
\hline \hline
 Suma& 0 \\ \hline
Resta& 1  \\ \hline
 And& 2  \\ \hline
 Or& 3  \\ \hline
 Not& 4  \\ \hline
 Xor& 5 \\ \hline
 Ca2& 6  \\ \hline
Shl& 7  \\ \hline

\end{tabular}
\caption{Tabla de seleccion de opcode} 
\label{tab=alu op}
\end{center}
\end{table}

La implementaci\'on en verilog no es acorde al esquema propuesto anteriormente debido a que no se pudo solucionar problemas de sincronizacion entre los distintos modulos. Se utilizo un m\'odulo para cada operacion.

\end{document}
