\documentclass[../../../informe/src/main.tex]{subfiles}

\begin{document}
\section{Ejercicio 5}

El formato BCD (de sus siglas en ingl\'es \textit{binary coded decimal}, es decir decimal codificado en binario) es aqu\'el que representa un d\'igito decimal por cada \textit{nybble} binario. Si bien esto es menos eficiente que la representaci\'on binaria en cuanto a que quedan 6 combinaciones de \textit{bits} sin usar por cada 16 (lo que normalmente ser\'ian los n\'umeros del 10 al 15), simplifica la conversi\'on de binario a decimal. A su vez, permite mayor precisi\'on en algunos casos: por ejemplo, el n\'umero $0.1_{10}$ no puede re presentarse de forma exacta en binario con ninguna cantidad de \textit{bits}, mientras que s\'i se puede con una convenci\'on adecuada de BCD.\par

En este ejercicio se implement\'o un sumador de n\'umeros BCD con dos entradas y dos salidas de un d\'igito cada una: dos n\'umeros del 1 al 9 en la entrada, y las decenas y unidades de su suma en la salida. Cada d\'igito consta de cuatro \textit{bits}. \par

Para expresar el resultado en BCD, se utiliz\'o un registro extra de 5 \textit{bits} para calcular la suma, de forma tal que el resultado siempre pueda quedar contenido en \'el (puesto que el m\'aximo \textit{input} es $9+9 = 18 < 31 = 2^5-1$, e incluso si tomamos casos no v\'alidos $15+15<31$). Si este resultado obtenido es menor que 10, se pone un cero en las decenas y la suma en las unidades. De lo contrario, las decenas valen 1, y las unidades, la suma menos 10.\par

A su vez, puesto que tener input de valores entre el 10 y el 15 no es v\'alido a pesar de que los 4 \textit{bits} lo permiten, si esta situaci\'on ocurre se indica error seteando en 1 todos los \textit{bits} de la salida, de forma tal que no pueda interpretarse como un n\'umero en BCD. \par

Se verific\'o que funcionara adecuadamente con un banco de pruebas que hace todas las combinaciones posibles de entradas, tanto v\'alidas como no v\'alidas.\par


\end{document}
